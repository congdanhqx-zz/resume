%%%%%%%%%%%%%%%%%%%%%%%%%%%% Document Setup %%%%%%%%%%%%%%%%%%%%%%%%%%%%

% require LuaTex

\documentclass[10pt]{article}
\usepackage{amssymb,latexsym}     % Standard packages
\usepackage{lastpage}
\usepackage{mathtools}
% for unicode character
\usepackage{fontspec}

% The automated optical recognition software used to digitize resume
% information works best with fonts that do not have serifs. This
% command uses a sans serif font throughout. Uncomment both lines (or at
% least the second) to restore a Roman font (i.e., a font with serifs).
%\usepackage{times}
\renewcommand{\familydefault}{\sfdefault}
\renewcommand\emph[1]{#1}

% This is a helpful package that puts math inside length specifications
\usepackage{calc}

% This package helps LaTeX auto-hyphenate hyphenated words if you use
% special hyphens. For example, bio\-/mimicry will properly hyphenate
% ``mimicry'' if necessary.
\usepackage[shortcuts]{extdash}

% Layout: Puts the section titles on left side of page
\reversemarginpar


%% Use these lines for A4-sized paper
\usepackage[paper=a4paper,
            %includefoot, % Uncomment to put page number above margin
            marginparwidth=30.5mm,    % Length of section titles
            marginparsep=1.5mm,       % Space between titles and text
            margin=25mm,              % 25mm margins
            includemp]{geometry}

%% More layout: Get rid of indenting throughout entire document
\setlength{\parindent}{0in}

% Provides special list environments and macros to create new ones
\usepackage[shortlabels]{enumitem}


\makeatletter
\newlength{\bibhang}
\setlength{\bibhang}{1em}
\newlength{\bibsep}
 {\@listi \global\bibsep\itemsep \global\advance\bibsep by\parsep}
\newlist{bibsection}{itemize}{3}
\setlist[bibsection]{label=,leftmargin=\bibhang,%
        itemindent=-\bibhang,
        itemsep=\bibsep,parsep=\z@,partopsep=0pt,
        topsep=0pt}
\newlist{bibenum}{enumerate}{3}
\setlist[bibenum]{label=[\arabic*],resume,leftmargin={\bibhang+\widthof{[999]}},%
        itemindent=-\bibhang,
        itemsep=\bibsep,parsep=\z@,partopsep=0pt,
        topsep=0pt}
\let\oldendbibenum\endbibenum
\def\endbibenum{\oldendbibenum\vspace{-.6\baselineskip}}
\let\oldendbibsection\endbibsection
\def\endbibsection{\oldendbibsection\vspace{-.6\baselineskip}}
\makeatother

%%% Setup header and footer (with page number and possible last page)
%
% The first block sets up pages 2--end
% The second block sets up page 1 formatting
%
%%%
%
% NOTE: comment the +LP lines and uncomment the -LP lines to have page
%       numbers without the ``of ##'' last page reference)
%
% NOTE: uncomment the \pagestyle{empty} line to get rid of all page
%       numbers on pages 2--end. To get rid of page numbers on page 1,
%       comment out the \thispagestyle{plain} line on the first page
%       below.
%       (also make sure includefoot is commented out above)
%
\usepackage{fancyhdr,lastpage,datetime}
\pagestyle{fancy}
\fancyhf{}\renewcommand{\headrulewidth}{0pt}
\fancyfootoffset{\marginparsep+\marginparwidth}
\newlength{\footpageshift}
\setlength{\footpageshift}
          {0.5\textwidth+0.5\marginparsep+0.5\marginparwidth-2in}

%%%% PAGES 2--9 NUMBERING:
%% These two lines put page number in upper-right corner of pages 2--end
\rhead{ĐOÀN Trần Công Danh, p.~\arabic{page} of \protect\pageref*{LastPage}}
\lhead{\newdateformat{rjh}{\monthname~\THEYEAR}}

%% These lines put page number in bottom (center) of pages 2--end
%\lfoot{\hspace{\footpageshift}%
%       \parbox{4in}{\, \hfill %
%                    \arabic{page} of \protect\pageref*{LastPage} % +LP
%%                    \arabic{page}                               % -LP
%                    \hfill \,}}
%%%% END PAGE 2--9 NUMBERING

%%%% PAGE 1 NUMBERING:
\makeatletter
\let\oldps@plain\ps@plain
\renewcommand{\ps@plain}{\oldps@plain%
\renewcommand{\@evenfoot}{\hspace*{-\footpageshift}\hfil %
    p.~\arabic{page} of \protect\pageref*{LastPage} % +LP
%    p.~\arabic{page}                               % -LP
    \hfil}%
\renewcommand{\@oddfoot}{\@evenfoot}}
\makeatother
%%%% END PAGE 1 NUMBERING

% Finally, give us PDF bookmarks and colored links
%
% NOTE: Some OCR software might be negatively affected by hyperlinks. So
%       most employers recommend the draft option here. Alternatively,
%       making all links black (as opposed to darkblue) should hopefully
%       prevent problems with most OCR.
%
% (to enable hyperlinks and bookmarks, comment out ``draft'' line;
%  to disable hyperlinks and bookmarks, uncomment ``draft'' line)
\usepackage{color,hyperref}
\definecolor{darkblue}{rgb}{0.0,0.0,0.3}
\hypersetup{breaklinks,colorlinks,
            linkcolor=black,urlcolor=black,
            anchorcolor=black,citecolor=black,
            %linkcolor=darkblue,urlcolor=darkblue,
            %anchorcolor=darkblue,citecolor=darkblue,
            %draft
            }

%%%%%%%%%%%%%%%%%%%%%%%% End Document Setup %%%%%%%%%%%%%%%%%%%%%%%%%%%%


%%%%%%%%%%%%%%%%%%%%%%%%%%% Helper Commands %%%%%%%%%%%%%%%%%%%%%%%%%%%%

%%% HEADING AT TOP OF CURRICULUM VITAE

% The title (name) with a horizontal rule under it
% (optional argument typesets an object right-justified across from name
%  as well)
%
% Usage: \makeheading{name}
%        OR
%        \makeheading[right_object]{name}
%
% Place at top of document. It should be the first thing.
% If ``right_object'' is provided in the square-braced optional
% argument, it will be right justified on the same line as ``name'' at
% the top of the CV. For example:
%
%       \makeheading[\emph{Curriculum vitae}]{Your Name}
%
% will put an emphasized ``Curriculum vitae'' at the top of the document
% as a title. Likewise, a picture could be included:
%
%   \makeheading[{\includegraphics[height=1.5in]{my_picture}}]{Your Name}
%
% the picture will be flush right across from the name. For this example
% to work, make sure the extra set of curly braces is included. Also
% makes ure that \usepackage{graphicx} is somewhere in the preamble.
\newcommand{\makeheading}[2][]%
        {\hspace*{-\marginparsep minus \marginparwidth}%
         \begin{minipage}[t]{\textwidth+\marginparwidth+\marginparsep}%
             {\large \bfseries #2 \hfill #1}\\[-0.15\baselineskip]%
                 \rule{\columnwidth}{1pt}%
         \end{minipage}}

%%% SECTION HEADINGS

% The section headings. Flush left in small caps down pseudo-margin.
%
% Usage: \section{section name}
\renewcommand{\section}[1]{\pagebreak[3]%
    \vspace{1.3\baselineskip}%
    \phantomsection\addcontentsline{toc}{section}{#1}%
    \noindent\llap{\scshape\smash{\parbox[t]{\marginparwidth}{\hyphenpenalty=10000\raggedright #1}}}%
%    \noindent\llap{\scshape\smash{\borderrightparbox{#1}}}%
    \vspace{-\baselineskip}\par\hyphenpenalty=10000}

\usepackage{tikz}
% horizontal rule between frames (using TikZ)
%\renewcommand{\ffvrule}[3]{%
%\hfill
%\tikz{%
%   \draw[loosely dotted,color=blue,line width=1.5pt,yshift=-#1] 
%   (0,0) -- (0pt,#3);}%
%\hfill\mbox{}}
%\insertvrule{flow}{1}{flow}{2}

%%% LISTS

% This macro alters a list by removing some of the space that follows the list
% (is used by lists below)
\newcommand*\fixendlist[1]{%
    \expandafter\let\csname preFixEndListend#1\expandafter\endcsname\csname end#1\endcsname
    \expandafter\def\csname end#1\endcsname{\csname preFixEndListend#1\endcsname\vspace{-0.6\baselineskip}}}

% These macros help ensure that items in outer-type lists do not get
% separated from the next line by a page break
% (they are used by lists below)
\let\originalItem\item
\newcommand*\fixouterlist[1]{%
    \expandafter\let\csname preFixOuterList#1\expandafter\endcsname\csname #1\endcsname
    \expandafter\def\csname #1\endcsname{\let\oldItem\item\def\item{\pagebreak[2]\oldItem}\csname preFixOuterList#1\endcsname}
    \expandafter\let\csname preFixOuterListend#1\expandafter\endcsname\csname end#1\endcsname
    \expandafter\def\csname end#1\endcsname{\let\item\oldItem\csname preFixOuterListend#1\endcsname}}
\newcommand*\fixinnerlist[1]{%
    \expandafter\let\csname preFixInnerList#1\expandafter\endcsname\csname #1\endcsname
    \expandafter\def\csname #1\endcsname{\let\oldItem\item\let\item\originalItem\csname preFixInnerList#1\endcsname}
    \expandafter\let\csname preFixInnerListend#1\expandafter\endcsname\csname end#1\endcsname
    \expandafter\def\csname end#1\endcsname{\csname preFixInnerListend#1\endcsname\let\item\oldItem}}

% An itemize-style list with lots of space between items
%
% Usage:
%   \begin{outerlist}
%       \item ...    % (or \item[] for no bullet)
%   \end{outerlist}
\newlist{outerlist}{itemize}{3}
    \setlist[outerlist]{label=\enskip\textbullet,leftmargin=*}
    \fixendlist{outerlist}
    \fixouterlist{outerlist}

% An environment IDENTICAL to outerlist that has better pre-list spacing
% when used as the first thing in a \section
%
% Usage:
%   \begin{lonelist}
%       \item ...    % (or \item[] for no bullet)
%   \end{lonelist}
\newlist{lonelist}{itemize}{3}
    \setlist[lonelist]{label=\enskip\textbullet,leftmargin=*,partopsep=0pt,topsep=0pt}
    \fixendlist{lonelist}
    \fixouterlist{lonelist}

% An itemize-style list with little space between items
%
% Usage:
%   \begin{innerlist}
%       \item ...    % (or \item[] for no bullet)
%   \end{innerlist}
\newlist{innerlist}{itemize}{3}
    \setlist[innerlist]{label=\enskip\textbullet,leftmargin=*,parsep=0pt,itemsep=0pt,topsep=0pt,partopsep=0pt}
    \fixinnerlist{innerlist}

% An environment IDENTICAL to innerlist that has better pre-list spacing
% when used as the first thing in a \section
%
% Usage:
%   \begin{loneinnerlist}
%       \item ...    % (or \item[] for no bullet)
%   \end{loneinnerlist}
\newlist{loneinnerlist}{itemize}{3}
    \setlist[loneinnerlist]{label=\enskip\textbullet,leftmargin=*,parsep=0pt,itemsep=0pt,topsep=0pt,partopsep=0pt}
    \fixendlist{loneinnerlist}
    \fixinnerlist{loneinnerlist}

%%% EXTRA SPACE

% To add some paragraph space between lines.
% This also tells LaTeX to preferably break a page on one of these gaps
% if there is a needed pagebreak nearby.
\newcommand{\blankline}{\quad\pagebreak[3]}
\newcommand{\halfblankline}{\quad\vspace{-0.5\baselineskip}\pagebreak[3]}

%%% FORMATTING MACROS

% Provides a linked \doi{#1} that links doi:#1 to http://dx.doi.org/#1
\usepackage{doi}
% To change the text before the DOI, adjust this command
%\renewcommand\doitext{doi:}

% Provides a linked \url{#1} that doesn't require escape characters
\usepackage{url}

% You can adjust the style \url{} uses here:
% (options are: same, rm, sf, tt; defaults to tt)
\urlstyle{same}

% For \email{ADDRESS}, links ADDRESS to the url mailto:ADDRESS
% (uncomment to typeset the e\-/mail address in typewriter font;
%  otherwise, will be typeset in the \urlstyle above)
%\DeclareUrlCommand\emaillink{\urlstyle{tt}}
\providecommand*\emaillink[1]{\nolinkurl{#1}}
\providecommand*\email[1]{\href{mailto:#1}{\emaillink{#1}}}
\providecommand*\skype[1]{\href{skype:#1}{#1}}

\providecommand\BibTeX{{B\kern-.05em{\sc i\kern-.025em b}\kern-.08em \TeX}}
\providecommand\Matlab{\textsc{Matlab}}

% Custom hyphenation rules for words that LaTeX has trouble with
\hyphenation{bio-mim-ic-ry bio-in-spi-ra-tion re-us-a-ble pro-vid-er Media-Wiki}

%%%%%%%%%%%%%%%%%%%%%%%% End Helper Commands %%%%%%%%%%%%%%%%%%%%%%%%%%%

%%%%%%%%%%%%%%%%%%%%%%%%% Begin CV Document %%%%%%%%%%%%%%%%%%%%%%%%%%%%

\begin{document}
\thispagestyle{plain}
\makeheading[\emph{Résumé}]{~DOAN~(Tran Cong)~Danh~}

\section{Contact Information}

% NOTE: Mind where the & separators and \\ breaks are in the following
%       table. Table is one row made up of three parboxes. The left
%       parbox has address info, the middle parbox has a vertical bar,
%       and the right parbox has phone and electronic contact
%       information.
%
% MACROS: \rcollength is the width of the right column of the table
%             (adjust it to your liking; default is 1.85in).
%         \spacewidth is width of area between left and right boxes.
%
\newlength{\rcollength}\setlength{\rcollength}{2.3in}%
\newlength{\spacewidth}\setlength{\spacewidth}{10pt}
%
\begin{tabular}[t]{@{}p{\textwidth-\rcollength-\spacewidth}@{}p{\spacewidth}@{}p{\rcollength}}%

% Address box
\parbox{\rcollength}{%
\emph{Name:} DOAN Tran Cong Danh\\
175/20 Vo Thanh Trang, Tan Binh\\
Ho Chi Minh city\\
Viet Nam
}

&
% Uncomment to add a vertical bar in middle of contact information
%{\vrule width 0.5pt}
\parbox[m][3\baselineskip]{\spacewidth}{}
&

% Non-snail-mail contact information
\parbox{\rcollength}{%
\emph{Native Name:} ĐOÀN Trần Công Danh\\
\emph{Skype:} \skype{congdanhqx}\\
\emph{Mobile:} +84-90-651-7911 \\
\emph{E-mail:} \email{congdanhqx@live.com}
}

\end{tabular}
\emph{Linked In:} \url{https://vn.linkedin.com/in/congdanhqx}\\
\emph{Blog:} \url{https://congdanhqx.github.io/}\\
\emph{Stack Overflow:} \url{https://stackoverflow.com/users/4115625/danh}

%%
%% In modern CV's, it seems like ``Objective'' is frowned upon. Instead,
%% incorporate it into a well-constructed cover letter. The ``More
%% information'' can go at the end of the CV, but it should not distract
%% from the section giving references available to contact.
%%
%
% \section{Objective}
%
% Full-time software engineer
% \begin{innerlist}
%     \item To be a professional engineer across the countries.
% \end{innerlist}

\section{Professional Summary}
\begin{loneinnerlist}
    \item Worked in parallel with another team in difference timezones.
    \item Have good experience in modern software process model.
    \item Have more than 4 years of strong experience with C$+$$+$ including C$+$$+$11.
\end{loneinnerlist}

\section{Professional Experience}
\href{http://www.luxoft.com/}{\textbf{Luxoft}}, Vietnam%
    \hfill \textbf{Nov 2015 to Present}
\begin{outerlist}
    \item[] \emph{HMI-DIS 2} Harman-Becker%
        \hfill \textbf{Nov 2015 to Present}
            \begin{innerlist}
                \item Used CoC HMI Framework (based on XML) to build
                     GUI for new component.
                \item Used MISRA-C++ to build the back-end of above GUI screens,
                     which runs on top of QNX OS.
            \end{innerlist}
\end{outerlist}

\blankline

\href{http://www.fpt-software.com/}{\textbf{FPT Software}}, Vietnam%
    \hfill \textbf{Nov 2013 to Oct 2015}
\begin{outerlist}
    \item[] \emph{Video Input Processing} DirecTV%
        \hfill \textbf{Nov 2013 to Oct 2015}
            \begin{innerlist}
                \item Worked in parallel with an California based team
                \item Used C$+$$+$11 and boost to develop a system to process
                     RTP stream which received from program provider (HBO, CNN,
                     etc...) by blacking out, filtering advertisement scene (to
                     replace it later), packetizing, encrypting (with AES-256),
                     recording in real-time to a format conformed with DASH
                     and/or HLS standard.
                \item Used HTML5, and JS to build some small webpage to display
                     current state of above system.
                \item Used C$+$$+$11, and boost to build a basic web-server
                     which provides the web-storage for those above webpages,
                     and current state of the system via a web API.
                \item Used Ruby to build a dummy cache server to reduce
                     workload for the above web-server (which ties to the Video
                     Processing System).
                \item Those systems would be run on top of Red Hat Enterprise
                     Linux 6.5 on a DELL R620 with Intel Xeon E5-2600 v2
                     processor (to take advantage of the builtin AES encryption
                     in that processor).
            \end{innerlist}
%    \item[] \emph{Traffic Monitoring} Government of Vietnam%
%        \hfill \textbf{May 2012 to Aug 2012}
%            \begin{innerlist}
%                \item Interned in a team uses MFC and C\# to develop a system to
%                     monitor vehicles come in and out of seaports.
%                \item That project is being applied to detect the violation of
%                     vehicles in the Nội Bài-Lào Cai expressway.
%            \end{innerlist}
\end{outerlist}

%\halfblankline

\section{Education}

\href{http://www.fpt.edu.vn/}{\textbf{FPT University}}, Vietnam
\begin{outerlist}
\item[] Bachelor of Software Engineering \hfill \textbf{September 2013}
%    \begin{innerlist}
%        \item Captone Project: Building hardware and firmware for a wireless
%        keyboard using MSP430 MCU and Nordic nRF24L01 RF transceiver.
%        \item Project Repository: https://bitbucket.org/congdanhqx/keyboard
%    \end{innerlist}
\end{outerlist}

\section{More Information}
\restartlist{bibenum}
Traveled (for pleasure and/or business) to:
\begin{innerlist}
    \item For business: the US, the Phillipines.
    \item For pleasure: Singapore, Cambodia, Myanmar.
\end{innerlist}

\halfblankline

Favourite quote:
\begin{innerlist}
    \item ``omne trium perfectum'' in Latina.
\end{innerlist}

\halfblankline

%Recently participate in Quora, Stack Overflow.

%\halfblankline

Open-sourced side project:
\begin{innerlist}
    \item 2 Steps Authenticator for BlackBerry 10
        \url{http://github.com/congdanhqx/gauth}
    \item Number Stringify
        \url{https://github.com/congdanhqx/numberstringify/}
    \item C++11 State Machine (in progress)
        \url{https://github.com/congdanhqx/state_machine}
\end{innerlist}

\vspace{0.1in}

\end{document}
